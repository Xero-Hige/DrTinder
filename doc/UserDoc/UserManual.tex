\documentclass[10pt,letterpaper,extrafontsizes]{memoir}
\listfiles
\usepackage{comment}
\usepackage{url,hyperref}
\usepackage[hang,small,labelfont=,textfont=]{caption}
\usepackage[spanish]{babel}
\usepackage[utf8]{inputenc}
\usepackage{shorttoc}
\usepackage{graphicx}
\usepackage{makeidx}
\usepackage{float}
\usepackage{tikz}
\usetikzlibrary{shapes,snakes}
\usepackage{amsmath,amssymb}

\newcommand{\MONTH}{%
  \ifcase\the\month
  \or Enero % 1
  \or Febrero % 2
  \or Marzo % 3
  \or Abril % 4
  \or Mayo % 5
  \or Junio % 6
  \or Julio % 7
  \or Agosto % 8
  \or Septiembre % 9
  \or Octubre % 10
  \or Noviembre % 11
  \or Diciembre % 12
  \fi}
 
%%%% Use the built-in division styling
\headstyles{memman}

%%% ToC down to subsections
\settocdepth{subsection}
%%% Numbering down to subsections as well
\setsecnumdepth{subsection}

%%%% extra index for first lines
\makeindex

% add patch to fink, such that \AtEndFile still work
\makeatletter
\AtEndFile{fink.sty}{
  \typeout{patching fink} 
  \renewcommand{\InputIfFileExists}[2]{%
    \IfFileExists{##1}%
    {##2\@addtofilelist{##1}%
      \m@matbeginf{##1}%
      \fink@prepare{##1}%
      %\@@input \@filef@und
      \expandafter\fink@input%
      \expandafter\fink@restore\expandafter{\finkpath}%
     \m@matendf{##1}%
     \killm@matf{##1}}%
 }
}
\makeatother

\makepagestyle{index}
  \makeheadrule{index}{\textwidth}{\normalrulethickness}
  \makeevenhead{index}{\rightmark}{}{\leftmark}
  \makeoddhead{index}{\rightmark}{}{\leftmark}
  \makeevenfoot{index}{\thepage}{}{}
  \makeoddfoot{index}{}{}{\thepage}

\makepagestyle{chapter}
\makeoddfoot{chapter}{}{\thepage}{}
\makeevenfoot{chapter}{}{\thepage}{}

%% end preamble
%%%%%%%%%%%%%%%%%%%%%%%%%%%%%%%%%%%%%%%%%%%%%%%%%%%%%%%
%#% extend

\usepackage[draft]{fixme}
\fxsetup{
  layout=marginnote
}
 

\usepackage[scaled=.92]{helvet}%. Sans serif - Helvetica
\usepackage{color,calc}
\newsavebox{\ChpNumBox}
\definecolor{ChapBlue}{RGB}{153, 0, 153}
\makeatletter
\newcommand*{\thickhrulefill}{%
\leavevmode\leaders\hrule height 1\p@ \hfill \kern \z@}
\newcommand*\BuildChpNum[2]{%
\begin{tabular}[t]{@{}c@{}}
\makebox[0pt][c]{#1\strut} \\[.5ex]
\colorbox{ChapBlue}{%
\rule[-10em]{0pt}{0pt}%
\rule{1ex}{0pt}\color{black}#2\strut
\rule{1ex}{0pt}}%
\end{tabular}}
\makechapterstyle{BlueBox}{%
\renewcommand{\chapnamefont}{\large\scshape}
\renewcommand{\chapnumfont}{\Huge\bfseries}
\renewcommand{\chaptitlefont}{\raggedright\Huge\bfseries}
\setlength{\beforechapskip}{20pt}
\setlength{\midchapskip}{26pt}
\setlength{\afterchapskip}{40pt}
\renewcommand{\printchaptername}{}
\renewcommand{\chapternamenum}{}
\renewcommand{\printchapternum}{%
\sbox{\ChpNumBox}{%
\BuildChpNum{\chapnamefont\@chapapp}%
{\chapnumfont\thechapter}}}
\renewcommand{\printchapternonum}{%
\sbox{\ChpNumBox}{%
\BuildChpNum{\chapnamefont\vphantom{\@chapapp}}%
{\chapnumfont\hphantom{\thechapter}}}}
\renewcommand{\afterchapternum}{}
\renewcommand{\printchaptertitle}[1]{%
\usebox{\ChpNumBox}\hfill
\parbox[t]{\hsize-\wd\ChpNumBox-1em}{%
\vspace{\midchapskip}%
\thickhrulefill\par
\chaptitlefont ##1\par}}%
}

\begin{document}
\chapterstyle{demo3}
\tightlists
\midsloppy
\raggedbottom


%%%%%%%%%%%%%%%%%%%%%%%%%%%%%%%%%%%%%%%%%%%%%%%%%%%%%%%

\frontmatter
\pagestyle{empty}


% half-title page
\vspace*{\fill}
\begin{adjustwidth}{1in}{1in}
\begin{flushleft}
\HUGE\sffamily
\end{flushleft}
\begin{center}
\HUGE\sffamily  Dr. Tinder
\end{center}
\begin{flushright}
\LARGE \ttfamily  \textit{Manual de usuario}\\
\end{flushright}
\end{adjustwidth}
\begin{flushright}
\includegraphics[width=.8\textwidth]{graficos/imagenes/logo}
\end{flushright}
\vspace*{\fill}
\cleardoublepage

% title page
\vspace*{\fill}
\begin{center}
\HUGE\textsf{Dr. Tinder}\par
\end{center}
\begin{center}
\HUGE\textsf{Guía del usuario}\par
\end{center}

\begin{center}
\Huge\textsf{ }\par
\end{center}
\begin{center}
\LARGE\textsf{Gaston A. Martinez}\par
\bigskip
\normalsize\textsf{ \MONTH - \the \year }\par
\normalsize\textsf{Correspondiente a la versión 1.0.0.0}\par
\medskip
\end{center}
\vspace*{\fill}
\clearpage

% copyright page
\begingroup
\footnotesize
\setlength{\parindent}{0pt}
\setlength{\parskip}{\baselineskip}
%%\ttfamily
\textcopyright{} \the \year - Gaston Alberto Martinez\\
All rights reserved

\begin{center}
\begin{tabular}{ll}
Primera edicion:                        & Junio 2016 \\
Version de la aplicación: & (1.0.0.0)
\end{tabular}
\end{center}

Ultima modificacion \today


\endgroup

\clearpage
\vspace*{\fill}

% ToC, etc
%%%\pagenumbering{roman}
\pagestyle{headings}
%%%%\pagestyle{Ruled}

\tableofcontents
\vspace*{\fill}
\clearpage
\listoffigures
\clearpage
\listoftables
\clearpage

\chapter{Prefacio}

    \textit{Tener que programar el Tinder es una buena excusa para usarlo.}\\

{\raggedleft{\scshape Gaston A. Martinez} \\ Buenos Aires, Argentina\\ Junio 2016\par}

%%%%%%%%%%%%%%%%%%%%%%%%%%%%%%%%%%%%%%%%%%%%%%%%%%%%%%%%%%%%%%%%%%%%%%%%%
\chapterstyle{BlueBox}
\mainmatter

\chapter{Información General} \label{chap:infogeneral}

\section{Sistema}

Dr.Tinder es una aplicación de encuentros desarrollada para el sistema operativo Android. En ella distintos usuarios pueden conocerse desde la comodidad de su celular.


\section{Requerimientos del dispositivo}

\begin{table}[H]
\centering
\begin{tabular}{|c|c|}
\hline
Sistema operativo&Android 4.4.4 (KitKat) o superior\\ \hline
Colectividad&Acceso a internet móvil\\ \hline
&Acceso a la ubicación\\
\hline
\end{tabular}
\caption{Requerimientos}
\label{table:reqs}
\end{table}

\section{Requerimientos}

Para poder crear una cuenta es necesario tener una dirección de correo electrónico valida. A la misma le llegaran notificaciones y le servirá para manejar su cuenta.

%%%%%%%%%%%%%%%%%%%%%%%%%%%%%%%%%%%%%%%%%%%%%%%%%%%%%%%%%%%%%%%%
\begin{center}
\tikzstyle{mybox} = [draw=blue, fill=green!20, very thick,
    rectangle, rounded corners, inner sep=10pt, inner ysep=20pt]
\tikzstyle{fancytitle} =[fill=blue, text=white]
\begin{tikzpicture}
\node [mybox] (box){%
    \begin{minipage}{\textwidth}
        \emph{Es importante que tenga acceso a la cuenta de correos con la que se registra. Sin eso, no podrá recuperar los datos en caso de perder la contraseña.}
    \end{minipage}
};
\node[fancytitle, right=10pt, rounded corners] at (box.north west){\textbf{Importante}};
\end{tikzpicture}
\end{center}
%
%%%%%%%%%%%%%%%%%%%%%%%%%%%%%%%%%%%%%%%%%%%%%%%%%%%%%%%%%%%


\chapter{Instalación} \label{chap:primerUso}

\section{Descarga de la aplicación}

La aplicación se puede descargar desde la pagina principal del proyecto \href{https://github.com/Xero-Hige/DrTinder}{https://github.com/Xero-Hige/DrTinder}:\\


\begin{figure}[H]
    \centering
\includegraphics[width=0.6\textwidth]{graficos/capturas/a}
    \caption{Pagina de descargas}
    \label{fig:downloadPage}
\end{figure}

\begin{figure}[H]
    \centering
\includegraphics[width=0.6\textwidth]{graficos/capturas/c}
    \caption{Dialogo de descarga}
    \label{fig:downloaddiag}
\end{figure}

Al hacer click en el vinculo (figura \ref{fig:downloadPage}) se descargara un archivo \texttt{apk} mediante el cual va a poder instalar la aplicación.

\begin{figure}[H]
    \centering
\includegraphics[width=0.6\textwidth]{graficos/capturas/d}
    \caption{Descarga iniciada del archivo}
    \label{fig:downloaddiag}
\end{figure}

\begin{figure}[H]
    \centering
\includegraphics[width=0.6\textwidth]{graficos/capturas/e}
    \caption{Descarga completada del archivo}
    \label{fig:downloaddone}
\end{figure}

\section{Instalación}

La instalación se inicia al tocar la notificación de descarga finalizada (Figura \ref{fig:downloaddone} desde el archivo descargado en la sección anterior. El sistema disparara automáticamente el proceso (Figura \ref{fig:downloadacpt}).

\begin{figure}[H]
    \centering
\includegraphics[width=0.6\textwidth]{graficos/capturas/b}
    \caption{Instalación y permisos}
    \label{fig:downloadacpt}
\end{figure}



\subsection{Problemas comunes de instalación}

Para que la instalación sea exitosa es necesario tener habilitada la opción \textit{Fuentes desconocidas} en la sección de \emph{Seguridad} de las opciones (Figura \ref{fig:segact}).

\begin{figure}[H]
    \centering
\includegraphics[width=0.6\textwidth]{graficos/capturas/f}
    \caption{Opciones de seguridad}
    \label{fig:seg}
\end{figure}

\begin{figure}[H]
    \centering
\includegraphics[width=0.6\textwidth]{graficos/capturas/g}
    \caption{Fuentes desconocidas activada}
    \label{fig:segact}
\end{figure}

%%%%%%%%%%%%%%%%%%%%%%%%%%%%%%%%%%%%%%%%%%%%%%%%%%%%%%%%%%%%%%%%
\begin{center}
\tikzstyle{mybox} = [draw=red, fill=red!20, very thick,
    rectangle, rounded corners, inner sep=10pt, inner ysep=20pt]
\tikzstyle{fancytitle} =[fill=red, text=white]
\begin{tikzpicture}
\node [mybox] (box){%
    \begin{minipage}{\textwidth}
        Las opciones de fuentes desconocidas están para preservar la integridad de su teléfono. Al habilitar esta opción puede instalar aplicaciones no verificadas por Google. Tenga en cuenta, que una vez habilitada esta opción, el sistema no va a informar sobre posibles aplicaciones no seguras que se intenten instalar. Una vez instalado \textbf{Dr. Tinder} des-habilite dicha opción por su seguridad.
    \end{minipage}
};
\node[fancytitle, right=10pt, rounded corners] at (box.north west){\textbf{Cuidado}};
\end{tikzpicture}
\end{center}
%
%%%%%%%%%%%%%%%%%%%%%%%%%%%%%%%%%%%%%%%%%%%%%%%%%%%%%%%%%%%

\chapter{Crear un usuario}

Para poder utilizar la aplicación es necesario crear un usuario para la misma. Los usuarios se crean en base a un \emph{email}, por lo que es necesario ingresarlo en la pantalla principal, para luego poder utilizar la opción de \emph{Sign Up} (Figura \ref{fig:createppal}).

%%%%%%%%%%%%%%%%%%%%%%%%%%%%%%%%%%%%%%%%%%%%%%%%%%%%%%%%%%%%%%%%
\begin{center}
\tikzstyle{mybox} = [draw=blue, fill=yellow!20, very thick,
    rectangle, rounded corners, inner sep=10pt, inner ysep=20pt]
\tikzstyle{fancytitle} =[fill=blue, text=white]
\begin{tikzpicture}
\node [mybox] (box){%
    \begin{minipage}{\textwidth}
        \emph{Si el email ya esta en uso, el sistema no permitirá crear un nuevo usuario con el mismo. En caso de que no le permita ingresar su cuenta personal, póngase en contacto con los administradores.}
    \end{minipage}
};
\node[fancytitle, right=10pt, rounded corners] at (box.north west){\textbf{Importante}};
\end{tikzpicture}
\end{center}
%
%%%%%%%%%%%%%%%%%%%%%%%%%%%%%%%%%%%%%%%%%%%%%%%%%%%%%%%%%%%

\begin{figure}[H]
    \centering
\includegraphics[width=0.6\textwidth]{graficos/capturas/h}
    \caption{Pantalla principal. Sign Up}
    \label{fig:createppal}
\end{figure}

Una vez que se seleccione esa opción, se pasara a la pantalla para llenar los datos (Figura \ref{fig:createppagl}) 

\begin{figure}[H]
    \centering
\includegraphics[width=0.6\textwidth]{graficos/capturas/i}
    \caption{Pantalla de creación (1)}
    \label{fig:createppagl}
\end{figure}

%%%%%%%%%%%%%%%%%%%%%%%%%%%%%%%%%%%%%%%%%%%%%%%%%%%%%%%%%%%%%%%%
\begin{center}
\tikzstyle{mybox} = [draw=blue, fill=yellow!20, very thick,
    rectangle, rounded corners, inner sep=10pt, inner ysep=20pt]
\tikzstyle{fancytitle} =[fill=blue, text=white]
\begin{tikzpicture}
\node [mybox] (box){%
    \begin{minipage}{\textwidth}
        \emph{Dependiendo del tamaño del dispositivo puede que alguna de las opciones no entren en pantalla. Las mismas se encuentran por debajo y son accesibles mediante scroll.}
    \end{minipage}
};
\node[fancytitle, right=10pt, rounded corners] at (box.north west){\textbf{Importante}};
\end{tikzpicture}
\end{center}
%
%%%%%%%%%%%%%%%%%%%%%%%%%%%%%%%%%%%%%%%%%%%%%%%%%%%%%%%%%%%

\section{Datos del usuario}

\begin{table}[H]
\centering
\begin{tabular}{|c|c|c|c|}
\hline
Campo&Descripción&Es obligatorio&Referencia\\
\hline \hline
Nombre&Nombre para mostrar del usuario&SI&1\\ \hline
Edad&Edad del usuario (Mayor de 18 años)&SI&2\\ \hline
Password&Password del usuario (Mayor a 6 caracteres)&SI&3\\ \hline
Sexo&Sexo del usuario&SI&4\\ \hline
Sexo de las parejas&Sexo de los candidatos&NO&5\\ \hline
Intereses&Intereses del usuario&NO&6\\ \hline
\end{tabular}
\caption{Información del usuario}
\label{table:campos}
\end{table}

\begin{figure}[H]
    \centering
\includegraphics[width=0.6\textwidth]{graficos/capturas/j}
    \caption{Pantalla de creación (2)}
    \label{fig:createppagl}
\end{figure}

Una vez completados los datos se puede proceder a crear el perfil presionando el boton \emph{create} (Figura \ref{fig:createppagl})

%%%%%%%%%%%%%%%%%%%%%%%%%%%%%%%%%%%%%%%%%%%%%%%%%%%%%%%%%%%%%%%%
\begin{center}
\tikzstyle{mybox} = [draw=blue, fill=yellow!20, very thick,
    rectangle, rounded corners, inner sep=10pt, inner ysep=20pt]
\tikzstyle{fancytitle} =[fill=blue, text=white]
\begin{tikzpicture}
\node [mybox] (box){%
    \begin{minipage}{\textwidth}
        \emph{Hasta que no se terminen de cargar los datos al servidor el usuario no estara disponible. Espere a que aparezca la notificacion de error de creacion o sea redirigido a la pantalla principal}
    \end{minipage}
};
\node[fancytitle, right=10pt, rounded corners] at (box.north west){\textbf{Importante}};
\end{tikzpicture}
\end{center}
%
%%%%%%%%%%%%%%%%%%%%%%%%%%%%%%%%%%%%%%%%%%%%%%%%%%%%%%%%%%%

\section{Subir una imagen}

Para poder subir la imagen de avatar se debe presionar la imagen por defecto (Figura \ref{fig:notfound}) y eso disparara la galería desde donde se podrá elegir la imagen a utilizar como avatar.

\begin{figure}[H]
    \centering
\includegraphics[width=0.6\textwidth]{graficos/imagenes/not_found}
    \caption{Imagen por defecto (imagen faltante)}
    \label{fig:notfound}
\end{figure}

%%%%%%%%%%%%%%%%%%%%%%%%%%%%%%%%%%%%%%%%%%%%%%%%%%%%%%%%%%%%%%%%
\begin{center}
\tikzstyle{mybox} = [draw=blue, fill=green!20, very thick,
    rectangle, rounded corners, inner sep=10pt, inner ysep=20pt]
\tikzstyle{fancytitle} =[fill=blue, text=white]
\begin{tikzpicture}
\node [mybox] (box){%
    \begin{minipage}{\textwidth}
        \emph{La imagen se cargara solo cuando se de presione el botón crear, por lo que se puede elegir múltiples veces la misma sin problema.}
    \end{minipage}
};
\node[fancytitle, right=10pt, rounded corners] at (box.north west){\textbf{A saber}};
\end{tikzpicture}
\end{center}
%
%%%%%%%%%%%%%%%%%%%%%%%%%%%%%%%%%%%%%%%%%%%%%%%%%%%%%%%%%%%
\section{Log in}

Una vez creado el usuario se puede loguear desde la pantalla principal (Figura \ref{fig:ppal}), utilizando el email y contraseña previamente registrados.

\begin{figure}[H]
    \centering
\includegraphics[width=0.6\textwidth]{graficos/capturas/k}
    \caption{Pantalla principal (Log in)}
    \label{fig:ppal}
\end{figure}


\chapter{Pantalla Principal}

\section{Pantalla de Selección}

\section{Pantalla de Chats}

\section{Cerrar Sesión}

\chapter{Buscar Matches}

\section{Selección de personas}

\section{Perfil de los usuarios}

\chapter{Chats}

\section{Visualización de Matches}

\section{Chat con un Match}

\chapter{Perfil}

\section{Actualizar los datos}

\section{Borrar el perfil}

\chapter{Errores comunes}

\section{Notificación de errores}

\backmatter

\pagestyle{index}
%\renewcommand{\chaptermark}[1]{}
\renewcommand{\preindexhook}{%
The first page number is usually, but not always, the primary reference to
the indexed topic.\vskip\onelineskip}
\indexintoc

%%%\raggedright  does nasty things to index entries
\printindex

\onecolindex

\pagestyle{empty}
\null\vfil

\begin{adjustwidth}{0in}{0in}
\chapter{Licencia y Copyright}


\addcontentsline{toc}{chapter}{Licencia y Copyright}

{\noindent
Copyright \copyright\ Gaston Alberto Martinez <gaston.martinez.90@gmail.com> \\
}

\begin{center}
\noindent
\includegraphics[height=1.5cm]{graficos/cc/cc}
\hspace{0.5cm}
\includegraphics[height=1.5cm]{graficos/cc/by}
\hspace{0.5cm}
\includegraphics[height=1.5cm]{graficos/cc/sa}
\end{center}

Esta obra se distribuye bajo la
\href{http://creativecommons.org/licenses/by-sa/4.0/deed.es}{Licencia Creative
Commons Atribución-CompartirIgual 4.0 Internacional}.

Los íconos utilizados fueron diseñados por
\href{http://www.freepik.com/}{Freepik}.

Este manual fue creado en LaTeX  basado en el modelo de Leslie Lamport y utilizando el template de \texttt{memoir:} \href{http://texdoc.net/texmf-dist/doc/latex/memoir/memman.tex}{\texttt{memman}}. 

\end{adjustwidth}

\vfil

\end{document}